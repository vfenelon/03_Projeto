\section{Calculos Básicos}
Porque está linha não esta junto com ooooooooooooooooooooooooooooooooooooooooooooooooooooooooooooooooooooooooooooooooooooooooooooooooooooooooooooooooooooooooooooooo

\subsection{Corrente de projeto Ip ou Ib}

Esquema de distribuição de condutores vivos:
Corrente alternada monofásico a dois condutores F + N (220V ou 127V)
Corrente alternada trifásico a quatro condutores 3F + N (380V ou 220V)

\subsection{Metodos para dimensionamento de cabos}
 
 Método de referência para os calculos de capacidade de condução de corrente
B1 
Condutores isolados ou cabos unipolares em eletroduto circulares aparaente ou embutido em alvenaria. Tipo de instalação mais comum, considerado para alimentação do QDP (quadro de distribuição do painel).
Esquema de aterramento
A alimentação do quadro de ser composto por uma proteção PE (aterramento), um neutro,  uma fase para monofásica ou três fases para trifásica
Em esquema TN o seccionaento automático visando proteção contra choque elétrico, pode ser usados dispositivos de proteção de sobrecorrente, dispositivos de proteção a corrente diferencial-residual (dispositivo DR), sendo este não é admitido quando usando a variante TN-C. 
Corrente de projeto: 
tabela
Circuito
Formula
Monofásico (F + N)

Trifásico (3F + N)
tabela com formula 
Sendo:

Pn – Potência elétrica nominal (W);
V – Tensão elétrica entre fase (V);
$\eta$ – Rendimento considerado = 1; 
cos $\varpi$ -  fator de potência (considerado = 1)
Numero de condutores carregados:
    • 2 para alimentação de quadro monofásico e circuitos de distribuição; 
    • 3 para alimentação de quadro trifásico.
Tabela 2: Capacidade de condução de corrente (A) para metodo B1; condutor de cobre, 70°C, PVC, temperatura ambiente 30°C(ar). Fonte tabela 36 NBR 5410:2004
começo tabela
fim tabela

    Parâmentros de projeto
        Corrente de curto circuito:   A
        Queda de tensão
        Fatores de demanda considerados
        Temperatura ambiente
    • Cálculo de demanda
    • Dimensionamento dos condutores
    • Dimensionamento dos eletrodutos
    • Dimensionamento das proteções
\subsubsection{ Metodo da queda de tensão}

\subsubsection{Dimensionamento Neutro e PE}

\subsubsection{ Metodo de instalação}

Dimensionamento de condutos
\subsection{ Proteção}

\subsubsection{Proteção contra Sobrecorrente (Sobrecarga e curto-circuito)}

Dispostivos de proteção que interrompa toda corrente de sobrecarga nos condutores dos circuitos antes que ela possa provocar aquecimento que prejudique a isolação, terminais ou vizinhanças das linha. Para isto deve satisfazer as seguintes condições

	$a^2 + b^2 = c^2$


Sendo:
Ip – corrente de projeto (A)
In – corrente nominal do dispositivo de proteção (A)
Ic -Capacidade de condução de corrente dos condutores vivos do circuito nas condições de instalação.
Iz –  Ic submetido a fatores de correção de agrupamento e ou fatores de correção de temperatura.
I2 – Corrente de atuação efetiva do dispositivo de proteção. 
acrescetar uma foto do disjuntor representando as grandezas das correntes

\subsubsection{ Disjuntores dimensionamento}
		
		METODO DE CALCULO DE CURTO-CIRCUITO 
		Coordenação e seletividade de disjuntores
\subsubsection{ Contra choques elétricos }
	6.1.2.1 Esquemas de aterramento
	6.1.2.2 IDR dimensionamento 

\subsubsection{ Proteção contra sobretensões}

\subsubsection{ Contra descarga atmosférica}

	Compatibilidade eletromagnética
\subsubsection{Quadros e acessórios}

    • Quadro de distribuição
        ◦ diagrama esquemático unifilar
        ◦ diagrama esquemático multifilar
    • Especificações dos componentes
    • Lista de materiais
Avisos devem ser colocados no quadro
foto do adesivo
Colocar aqui os adesivos e criar um anexo para a impressão dos adesivos.
Colocar os tamanhos de quadro metalicos disponiveis, considerados
ATENÇÃO A MANUTENÇÃO DO QUADRO ELETRICO DEVE SER REALIZADO POR PESSOAL HABILITADO. RISCO DE CHOQUE.