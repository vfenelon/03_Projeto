\section{Especificações Técnicas}

\subsection{Condutores elétricos}
Os cabos elétricos utilizados nas instalações devem possuir classe de encordoamento no mínimo 5 ou 6, tanto para os circuitos terminais, quanto para o alimentador do quadro de distribuição QGD1, os cabos dos circuitos terminais deverão ser cabos isolados de PVC (450/750V), nas cores preta, azul, vermelho, branco e ou verde) e unipolar em EPR para o alimentador do quando QGD1. 
Todos os condutores deverão seguir a NBR NM 247-3 e possuir o selo do inmetro, todas as seções desses condutores estão descritas na tabela de carga de acordo com os seus respectivos circuitos e deverão ser seguidos.
ATENÇÃO A escolha dos cabos devem ser realizadas conforme a Norma 5410.

\subsection{Disjuntores}
Os disjuntores devem ser de tensão nominal de serviço Ue = 400V ou superior. Seguir as correntes nominais de atuação de sobrecarga conforme indicado em projeto. As correntes de ruptura devem ser de 3kA para os disjuntores de circuitos terminais e 5kA para o disjuntor geral. As curvas de disparo devem seguir conforme indicado na tabela de carga, assim como a escolha da quantidade de pólos.
Todos os disjuntores existentes no projeto devem ser de fabrincantes que possuem o selo do INMETRO de certificação compulsória e os responsáveis pela compra e execução dos materiais deverão utilizadas marcas de confiança e consolidadas no mercado como Schneider, WEG, ABB, Siemens, Steck ou similar. Todos os disjuntores deverão seguir as normas NBR NM 60898 ou NBR IEC 60947-2 quando aplicável.  NBR 60898:2004 – Disjuntores para proteção de sobrecorrentes para instalações domésticas e similares 
NBR 60947-2 – Dispositivo de manobra e comando de baixa tensão

\subsection{IDR – Interruptor diferencial residual}

Dispostivos DR são seccionadores mecânicos projetados para provocar abertura na ocorrência de uma corrente de fuga à terrra. Tem como principal objetivo proteger as pessoas contra os efeitos dos choques elétricos prejudiciais a saúde. De acordo com NBR 5410/2004 item 5.1.3.2.2, seu uso é obrigatório nos circuitos elétricos localizados em áreas molhadas e externa [Siemens 2023].
Sua versão de corrente residual até 30 mA são destinados a principalmente a proteção de pessoas, acima deste valor, são apropriados para a proteção de instalações elétricas. Existe 3 tipos de dispositivos DR:
    1. Tipo AC – detecta correntes residuais alternadas;
    2. Tipo A – detecta correntes residuais alternadas e continua pulsantes.
    3. Tipo B – detecta correntes residuais alternadas, continuas pulsantes e continuas puras[Siemens 2023].
O IDR a ser utilizado deverá ser ligado em série com o disjuntor geral e este deve possuir corrente nominal igual ou superior a corrente nominal do disjuntor termomagnético geral do quadro QDC. O valor de atuação da corrente diferencial do IDR deverá ser de 30mA. O IDR deverá seguir a norma ABNT NBR NM 61008-1 (verificar validade no site da ABNT)
\subsection{DPS – Dispositivo de proteção contra surtos elétricos}
O DPS utilizado deverá ser do tipo classe II, 175V e 20kA e este deverá ser instalado dentro do quadro de distribuição QGD1. O DPS deverá seguir a norma ABNT NBR IEC 61643-1 (verificar validade no site da ABNT), NBR 5410 e NBR 5419.

\subsection{Quadro elétrico e acessórios}
    1 O quadro de distribuição deve seguir a norma NBR IEC 60439-1 e NBR 5410.

No quadro deverão estar presentes os seguintes itens:

    2 Tensão nominal: (220/127V)
    3 Corrente de demanda: 34,55A 
    4 Capacidade de Curto-circuito: 5kA
    5 Grau de proteção IP adequado e no mínimo IP2X.
    6 Placa de Identificação do quadro contendo nome ou marca do fabricante e tipo ou numero de identificação.

O instalador deverá inserir etiqueta de advertência conforme NBR 5410 no item 6.5.4.10 mostrada abaixo:

5.6.5.2 Quando existir risco de choque elétrico, o dispositivo de seccionamento de emergência deve seccionar todos os condutores vivos, observada a prescrição de 5.6.2.2
Perfil 
Canaleta DN
as canaletas DN servem para agrupar, proteger e organizar os fios e cabos dentro do quadro elétrico. São de PVC rígido em conformidade com a Norma Diretiva 2002/95/EC-RoHS